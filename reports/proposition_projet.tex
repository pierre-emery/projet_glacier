\documentclass[11pt]{article}
\usepackage[margin=1in]{geometry}
\usepackage{hyperref}
\usepackage{microtype}
\usepackage{xurl}
\usepackage{hyperref}

\title{Présentation de Projet : Prédire l'évolution des glaciers}
\author{Fatou Ndao 20235568 \and Mohamed Atmani 20218934 \and Pierre Emery 20278920}
\date{7 février 2026}

\begin{document}
\maketitle
\section*{Motivation}
Les glaciers reculent rapidement sous l'effet du réchauffement climatique, avec des impacts sur les ressources en eau (eau potable, irrigation, hydroélectricité), les écosystèmes de montagne et certains risques naturels (lacs glaciaires, crues, instabilités). Anticiper leur évolution est donc un enjeu concret d'adaptation. L'imagerie satellitaire permet d'observer ces changements de manière régulière et à grande échelle, ce qui ouvre la voie à des approches d'apprentissage automatique capables de modéliser et prédire l'évolution de l'étendue des glaciers.


\section*{Objectifs}
L'objectif de ce projet est de développer un modèle prédictif capable d'estimer l'évolution future de l'étendue d'un glacier à partir d'observations passées. Nous formulerons la tâche comme une prédiction supervisée : à partir de l'état d'un glacier à une date $t$ (image Sentinel-2 et informations issues des contours historiques), prédire son contour à une date future $t+\Delta t$. Cette formulation permet d'obtenir une sortie directement interprétable (un nouveau contour), à partir de laquelle on peut aussi dériver des mesures comme l'aire.

Nous comparerons plusieurs architectures convolutionnelles. Dans un premier temps, nous établirons un baseline peu coûteux en calcul (p.\,ex. All-CNN) pour valider rapidement l'approche. Ensuite, nous utiliserons des architectures plus adaptées à la tache, notamment U-Net. Nous pourrons également explorer l'usage d'une architecture plus profonde ou large, (comme Wide ResNet ou Resnet), comme encodeur au sein du modèle afin d'améliorer la qualité de la segmentation.


\section*{Jeux de données choisis et justification}
Pour ce projet, nous utiliserons principalement deux jeux de données :
\begin{itemize}
    \item \textbf{GLIMS Glacier Database, Version 1 (NSIDC-0272)} : Ce jeu de données fournit des contours datés pour de nombreux glaciers, permettant ainsi de construire des séries temporelles et d'analyser l'évolution des glaciers au fil du temps.
    \item \textbf{Sentinel-2} : Ce sont des images ayant une résolution spatiale élevée et nous permettant de détecter des détails supplémentaires sur les glaciers.
\end{itemize}

\vspace{0.8em}
\noindent\textbf{Liens datasets :}
\href{https://nsidc.org/data/nsidc-0272/versions/1}{GLIMS (NSIDC-0272)}
\quad -- \quad
\href{https://dataspace.copernicus.eu/data-collections/copernicus-sentinel-data/sentinel-2}{Sentinel-2 (Copernicus Data Space)}

\end{document}